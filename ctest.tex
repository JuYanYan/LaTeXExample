% !Mode::"TeX:UTF-8"
\documentclass[twocolumn,landscape,UTF8]{ctexart}
\usepackage{lastpage}
\usepackage{color}
\usepackage{ulem}
\usepackage{titlesec}
\usepackage{graphicx}
\usepackage{colortbl}
\usepackage{listings}
\usepackage{makecell}
\usepackage{indentfirst}
\usepackage{fancyhdr}
\usepackage{setspace}                          % 行间距
\usepackage{bm}                                % \boldsymbol 粗体
% 数学
\usepackage{amsmath,amsfonts,amsmath,amssymb,times}
\usepackage{txfonts}
\usepackage{enumerate}                         % 编号
\usepackage{tikz,pgfplots}                     % 绘图
\usepackage{tkz-euclide,pgfplots}
\usetikzlibrary{automata,positioning}
\usepackage[paperwidth=36.8cm,paperheight=26cm,top=2.5cm,bottom=2cm,right=2cm]{geometry}
\lstset{language=C,keywordstyle=\color{red},showstringspaces=false,rulesepcolor=\color{green}}
\oddsidemargin=0.5cm                           % 奇数页页边距
\evensidemargin=0.5cm                          % 偶数页页边距
\textwidth=30cm                                % 文本的宽度 单页

\newsavebox{\zdx}                              % 装订线

\newcommand{\putzdx}{\marginpar{
		\parbox{1cm}{\vspace{-1.6cm}
			\rotatebox[origin=c]{90}{
				\usebox{\zdx}
		}}
}}

\newcommand{\blank}{\uline{\textcolor{white}{a}\ \textcolor{white}{a}\ \textcolor{white}{a}\ \textcolor{white}{a}\ \textcolor{white}{a}\ \textcolor{white}{a}\ \textcolor{white}{a}\ \textcolor{white}{a}\ \textcolor{white}{a}\ \textcolor{white}{a}\ \textcolor{white}{a}}}

\newcommand{\me}{\mathrm{e}}                   % 定义 对数常数e,虚数符号i,j以及微分算子d为直立体。
\newcommand{\mi}{\mathrm{i}}
\newcommand{\mj}{\mathrm{j}}
\newcommand{\dif}{\mathrm{d}}
\newcommand{\bs}{\boldsymbol}                  % 数学黑体
\newcommand{\ds}{\displaystyle}
%通常我们使用的分数线是系统自己定义的分数线,即分数线的长度的预设值是分子或分母所占的最大宽度,如何让分数线的长度变长成,我们%可以在分子分母添加间隔来实现。如中文分式的命令可以定义为:
%\newcommand{\chfrac[2]}{\cfrac{\;#1\;}{\;#2\;}}
%\frac{1}{2} \qquad \chfrac{1}{2}

% 选择题
\newcommand{\fourch}[4]{\\\begin{tabular}{*{4}{@{}p{3.5cm}}}(A)~#1 & (B)~#2 & (C)~#3 & (D)~#4\end{tabular}} % 四行
\newcommand{\twoch}[4]{\\\begin{tabular}{*{2}{@{}p{7cm}}}(A)~#1 & (B)~#2\end{tabular}\\\begin{tabular}{*{2}{@{}p{7cm}}}(C)~#3 &
		(D)~#4\end{tabular}}                      % 两行
\newcommand{\onech}[4]{\\(A)~#1 \\ (B)~#2 \\ (C)~#3 \\ (D)~#4}  % 一行

\renewcommand{\headrulewidth}{0pt}
\pagestyle{fancy}
\begin{document}
\fancyhf{}
\fancyfoot[CO,CE]{\vspace*{1mm}第\,\thepage\,页 , 共 ~\pageref{LastPage} 页}
\sbox{\zdx}
{\parbox{27cm}{\centering
	座位号~\underline{\makebox[34mm][c]{}}~ 班~级\underline{\makebox[34mm][c]{}}~\CJKfamily{song} 学~号\underline{\makebox[44mm][c]{}}~\CJKfamily{song} 姓~名\underline{\makebox[34mm][c]{}} ~\\
	\vspace{3mm}
请直接在本试卷上作答。并填写试卷序号、班级、学号和 姓名\\
%答题时学号
\vspace{1mm}
\dotfill{} 密\dotfill{}封\dotfill{}线\dotfill{} \\
	}}
	\reversemarginpar
	
\begin{spacing}{1.25}
	 \begin{center}
        \begin{LARGE}
            期中考试试卷    \\
            
        \end{LARGE}
        (闭卷笔试\ \ 60 分钟)                                                \\
        \vspace{0.5cm}
        \begin{tabular}{|m{0.03\textwidth}|*{8}{m{0.035\textwidth}|}p{0.04\textwidth}|}
	      \hline
           \centering  题~号 & \centering 一 & \centering 二 & \centering 三 & \centering 四& \centering 五
         & \centering 总~分 & \makecell{阅卷人} \rule{0pt}{3mm}              \\
	      \hline
           \centering 分~数 &  &  &  &  &  &  &
           \rule{0pt}{8mm}                                                   \\
	      \hline
        \end{tabular}
    \end{center}
\end{spacing}
\vspace{-0.5cm}
\setlength{\marginparsep}{1.7cm}
\putzdx                                           % 装订线--奇页数
\vspace{1cm}
\begin{spacing}{1.3}
    \section*{\hspace{5cm} 一、不定项选择题~(每题~4 分)}
    \vspace{-2cm}
    \begin{tabular}{|p{0.05\textwidth}|p{0.05\textwidth}|}
			\hline
			% after \\: \hline or \cline{col1-col2} \cline{col3-col4} ...
			\centering 阅卷人& \\
			\hline
			\centering 得~~分 &  \\
			\hline
		\end{tabular}
\begin{enumerate}
\setcounter{enumi}{0}
    % 
    \item 下列理解\textbf{错误}的是 \ \ (\qquad)
    \onech{编译过程是一种翻译过程}{C语言程序不需要编译器或解释器即可被计算机执行}{C语言程序可以直接被CPU执行}{链接过程晚于编译过程}
    % 
    \item 对于语法错误、语义错误,理解\textbf{正确}的是 \ \ (\qquad)
    \twoch{语法错误可以被链接器发现}{语义错误是上下文无关的}{语法错误是由编译器发现的}{语义错误不是上下文无关的}
    % 
    \item 对于基本数据类型int,关于在\textbf{任何机器上}的情况,理解\textbf{错误}的是 \ \ (\qquad)
    \twoch{int一定是8位有符号整数类型}{int一定是16位有符号整数类型}{int一定是32位有符号整数类型}{int一定是64位有符号整数类型}
    % 
    \item 对于printf函数的格式化字符串(即\% d等),理解\textbf{正确}的是 \ \ (\qquad)
    \twoch{\%d对应着int类型的数据}{\%f对应着float类型的数据}{\%f对应着long类型的数据}{\%c对应着char类型的数据}
    %
    \item 对于一个C语言程序,理解\textbf{正确}的是 \ \ (\qquad)
    \onech{默认情况下程序将从main函数开始执行}{默认情况下程序将从mian函数开始执行}{主函数的返回值为int类型}{主函数的返回值为float类型}
    %
    \item 对于数据类型的理解,\textbf{正确}的是 \ \ (\qquad)
    \twoch{int在其可表示范围内是精确的}{char在其可表示范围内是精确的}{float可以表示整数}{double可以精确表示0.0这个数}
    %
    \item 用scanf输入当前作用域下可访问的int类型变量a, b,下列写法\textbf{可用}的是 \ \ (\qquad)
    \twoch{scanf(``\%d\%d, \&a, \&b'');}{scanf(``\%d\%d'', \&a, \&b);}
          {scanf(``\%d,\%d'', a, b);}{scanf(``\%d,\%d'', \&a, \&b);}
    %
    \item 已知char类型(可看作\textbf{有符号8位整数}类型)的变量a,内存中的二进制值为11111111,那么printf(``\%d'', (int)a)将得到 \ \ (\qquad)
    \fourch{-1}{?}{0}{(null)}
    %
    \item 下列\textbf{正确}的字面常量写法是 \ \ (\qquad)
    \fourch{012}{0ff}{0.32f}{``string string''}
\end{enumerate}
\section*{\hspace{4.5cm} 二、判断题:~正确~$\surd$, 错误~$\times$ (每题~3 分)}
\vspace{-1.5cm}
\begin{tabular}{|p{0.05\textwidth}|p{0.05\textwidth}|}
  \hline
    \centering 阅卷人&       \\
  \hline
    \centering 得~~分 &      \\
  \hline
\end{tabular}
\begin{enumerate}
\setcounter{enumi}{9}
    %
    \item 0xff是合法的十六进制值                                 \hfill(\qquad)
    %
    \item $-\frac{w}{2}$在C语言中可写为-w / 2.0;                \hfill(\qquad)
    %
    \item 表达式3 + 2 / 5的结果为3                              \hfill(\qquad)
    %
    \item 表达式1 + 2 > 3 == 4的结果为假                         \hfill(\qquad)
    % 
    \item	 $\text{unsigned int a} = -1;$则a的值至少为$2^{16}-1$  \hfill(\qquad)
\end{enumerate}
\section*{\hspace{4.5cm} 三、兰伯特余弦定律~(每空~3 分,共~15 分)}
\vspace{-1.5cm}
\begin{tabular}{|p{0.05\textwidth}|p{0.05\textwidth}|}
  \hline
    \centering 阅卷人 &         \\
  \hline
    \centering 得~~分 &         \\
  \hline
\end{tabular}
\begin{enumerate}\setcounter{enumi}{14}
    \item 下面程序用于模拟兰伯特余弦定律,即表面辐照度与光方向和表面法线夹角的余弦值成正比。输入该程序的包括光线方向向量$\vec{i}=\left(x,y,z\right)$、辐射通量$\boldsymbol{\Phi}$和面积$A$。法线$\vec n$题目中已给出。你需要按照公式$\boldsymbol{E}=\frac{\boldsymbol{\Phi}}{A}\cos{\theta}$计算出表面辐照度$\boldsymbol{E}$,并输出$\boldsymbol{E}$的值。为了方便起见,可以确定向量$\vec n$模长一定是1,同时我对$\vec i$做了归一化,这样它的模长也是1,如此就可用$\cos{\theta}=\vec{i}\cdot\vec{n}$计算夹角的余弦值,并且,你可以假定输入是合法的。
\begin{lstlisting}
输入样例1: 1 2 3 4 5
输出样例1: E = 0.8
输入样例2: 1 0 0 1 1
输出样例2: E = 0.267261
输入样例3: 2 4 6 1 1
输出样例3: E = 1
\end{lstlisting}
    例如对于样例1,它意味着$\vec{i}=\left(1,2,3\right)$,辐射通量$\boldsymbol{\Phi}=4$,面积$A=5$。请注意这个样例,输出的不是``E = 0.800000'',而是``E = 0.8''。程序我已经写了一部分,你只需要把程序补全即可。不过你完全可以写一个新的在边上,当然,要能通过样例数据。
\begin{lstlisting}
#include <math.h>
#include <stdio.h>
// 表面法线n=(1, 2, 3)的归一化结果, 模长为1
const float nX = 0.2672612419124244f;
const float nY = 0.5345224838248488f;
const float nZ = 0.8017837257372732f;
int main(void)
{
    float phi, A, cosTheta, x, y, z, leng, E;
    printf("format: x y z \\phi A\n");// 输入向量i是(x, y, z)
    //
    scanf(______________, &x, &y, &z, &phi, &A);
    leng = sqrt(x*x + y*y + z*z);     // 计算输入向量模长
    x /= leng;                        // 归一化输入向量
    y /= leng;
    z /= leng;
    cosTheta = ____________________;  // 用点乘计算余弦值
    E = (________) * ________;        // 用公式计算E
    printf(______________, E);
    return 0;
}
\end{lstlisting}
\end{enumerate}
\section*{\hspace{4.5cm} 四、张三买笔~(本题~20 分)}
\vspace{-1.5cm}
\begin{tabular}{|p{0.05\textwidth}|p{0.05\textwidth}|}
  \hline
    \centering 阅卷人 &         \\
  \hline
    \centering 得~~分 &         \\
  \hline
\end{tabular}
\begin{enumerate}\setcounter{enumi}{15}
    \item 张三走上了帮人跑腿的路。现在他遇到了一个小学生,小学生给了他a元b角,让他去校门外的奸商处购买签字笔。一只签字笔是2元3角,现在张三想知道自己能买多少签字笔回来。                                            \\
    张三当然有跑腿费,费用是1元1角。小学生给的钱自然也大于跑腿费。           \\
    输入格式:                                                        \\
    输入只有一行,两个整数,分别表示a和b。并且,它们不会超过你这个月的生活费  \\
    输出格式:                                                        \\
    输出只有一行,表示张三能买多少笔。按照我们的约定,这个结果大于等于0       \\
    样例输入:                                                        \\
    3 4                                                              \\
    样例输出:                                                        \\
    1                                                                \\
    因为万事开头难,所以我帮你把最开始的部分做了。
\begin{lstlisting}
#include <stdio.h>
int main(void)
{
    int a, b;


    scanf(                       );






    return 0;
}
\end{lstlisting}
\end{enumerate}
\section*{\hspace{4.5cm} 五、旦总的夜宵~(本题~14 分)}
\vspace{-1.5cm}
\begin{tabular}{|p{0.05\textwidth}|p{0.05\textwidth}|}
  \hline
    \centering 阅卷人 &         \\
  \hline
    \centering 得~~分 &         \\
  \hline
\end{tabular}
\begin{enumerate}\setcounter{enumi}{16}
    \item 旦总单纯地饿了,想点外卖吃披萨。为了让披萨看起来多一点,他想知道对着比萨切n刀,最多能切出多少块披萨? \\
    很显然,切太多了没法吃(因为切太碎了),因此输入数据就是个位数。\\
    输入只有一行,表示n。输出只有一行,表示能切出多少披萨。       \\
    输入样例:                                              \\
    3                                                      \\
    输出样例:                                              \\
    7                                                      \\
    我帮你把最开始的部分做了,考虑到大家的最终成绩,因此本题分值不高,祝你好运。
\begin{lstlisting}
#include <stdio.h>
int main(void)
{




    return 0;
}
\end{lstlisting}
\end{enumerate}

\end{spacing}

\clearpage
	
\end{document}
